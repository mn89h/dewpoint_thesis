\chapter{Conclusion}
In concluding the dew point hygrometer project, the concept of dew point detection is charming, providing a visible and understandable measure of humidity. In realization however, the concept proves hard to be consequently achieved. Despite the project's inability to meet its initial goals, the experience has provided valuable insights into the complexities of designing and implementing a reliable dew point detection system. The primary difficulties encountered in this endeavor were rooted in data analysis, time constraints, and the inherent challenges of accurately detecting dew points.

While the sensors provide acceptably good readings for a change in humidity and presence of dew, the main challenge lies in the control of this complex system. The temperature ranges vary vastly, needing proper consideration for temperature gradients, cooling thresholds and periodic measurements. Even then, without properly controlled conditions, statements on the achievable accuracy are difficult to make due to changes in temperature, also induced by self-heating, and in humidity over long measurement periods.

The data readout proved to be the main and limiting challenge however. It requires careful study and analysis of correlations and even more consideration in terms of using the correct measure and offset in order to detect an accurate value for the dew point temperature.

Despite these challenges, the project has laid a foundation for future research and development in the field of dew point hygrometry. The difficulties encountered in data analysis and the constraints imposed by time have highlighted the need for innovative approaches to sensor design, data interpretation, and experimental methodology. The exploration of new materials and better control and detection algorithms, as well as the refinement of the used ones, remains a goal for future work.
